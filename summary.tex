\documentclass{llncs}

\usepackage{latexsym}
\usepackage{amssymb}
\usepackage{amsmath}
\usepackage{pstricks}
%\usepackage[dvips]{pstricks}
%\usepackage{proof}

\date{}
\date{Version of \today}

% \newenvironment{lined}[1]%
%   {\begin{center}\begin{minipage}{#1}\hrule\medskip}
%   {\vspace{-1ex}\hrule \end{minipage}\end{center}}
% %% end of [newenvironment]
% 
% \def\lshfit#1#2{\kern-#1 #2\kern#1}
% \def\casebox{\psframebox[shadow=true,shadowsize=2pt,shadowangle=315]}
% \def\fillsquare{\kern2pt\raise0.25pt
%   \hbox{$\vcenter{\hrule height0pt \hbox{\vrule width5pt height5pt} \hrule height0pt}$}
% } %% end of [\def]
% 
% \def\ATS{{\cal A\kern-1ptT\kern-2ptS}}

\newcommand{\csppre}[2]{#1 \rightarrow #2}
\newcommand{\cspint}[2]{#1\ \sqcap\ #2}
\newcommand{\cspext}[2]{#1\ \Box\ #2}
\newcommand{\cspseq}[2]{#1; #2}
% \newcommand{\csppara}[3]{#1\ |_{#2}|\ #3}
\newcommand{\csppara}[3]{#1 \underset{#2}{\parallel} #3}
\newcommand{\csphide}[2]{#1\ \backslash\ #2}
\newcommand{\csptick}{\surd}
\newcommand{\csptau}{\tau}
\newcommand{\csptracesmodel}{\mathcal{T}}
\newcommand{\cspfdrmodel}{\mathcal{N}}
\newcommand{\cspfailuresmodel}{\mathcal{F}}
\newcommand{\cspm}{CSP$_M$}
\newcommand{\cspbold}{\textbf{CSP}}
\newcommand{\csp}{CSP}
\newcommand{\csps}{CSP\#}

\newcommand{\dmllampat}{$\lambda_{pat}$}
\newcommand{\dmlL}{$\mathcal{L}$}
\newcommand{\dmllamall}{$\lambda_{pat}^{\Pi,\Sigma}$}
\newcommand{\dmlzero}{$DML_0$}

\newcommand{\fdr}{FDR}
\newcommand{\pat}{PAT}
\newcommand{\prob}{PRoB}


%%%%%%%%%%%%%%%%%%%%%%%%%%%%%%%%%%%%%%%%%%%%%%
\pagestyle{plain}

% make a proper TOC despite llncs
\setcounter{tocdepth}{2}
\makeatletter
\renewcommand*\l@author[2]{}
\renewcommand*\l@title[2]{}
\makeatletter

\begin{document}

\title{%
Summary of Various Topics in \break
Type Sytem, Linear-Time Temporal Logic and Model Checking
} %% end of [\title]
\author{Zhiqiang Ren}
\institute{Boston University}

\maketitle % typeset the title of the contribution

% \begin{abstract}
% 
% Content of abstract.
% 
% \end{abstract}

% \keywords{type theory, dependent types, linear timed logic, model checking}

%%
\setcounter{page}{1}
%%

\tableofcontents 

\baselineskip=11.875pt

\newpage 
\section{Summary of ``Dependent ML: An Approach to Practical
Programming with Dependent Types''\cite{Xi2007Dependent}}
  \label{section:DML}

This paper presents an approach to applying dependent types in practical
programming. Taking such approach, users write programs in a functional
programming language (Dependent ML), which allows for specification and
inference of more precise type information than other common programming
languages. Such precise type information can then be used to facilitate
program error detection and compiler optimization.

In summary, the whole system involves four languages including \dmllampat, \dmlL,
\dmllamall{} and \dmlzero. \dmlzero{} is the outmost part of the system, in which users
write program. \dmllamall{} is an intermediate language with complex syntax,
from which precise type information can be extracted. Language \dmlL{} is for
building type index used by both \dmlzero{} and \dmllamall. Language
\dmllampat{} serves as a bridge between \dmlzero{} and \dmllamall, representing the dynamic
semantics of expressions in the later two languages. With the aforementioned
settings, the paper gives out elaboration rules to map expression in
\dmlzero{} into expression in \dmllamall{} which conveys more precise types, while
maintaining the dynamic semantics. A detailed illustration of each of these
four languages and their relations goes as follows.

Firstly, \dmllampat{} is a simply typed language, which extends the
simply typed $\lambda$-calculus with recursion and general pattern matching. The
dynamic semantics of expression in \dmllampat{} is assigned through the use of
evaluation context and redex. The proof of type soundness of \dmllampat{} is
given to show the constraint on the result of the evaluation of a well-typed
expression. A sensible \emph{operational equivalence} relation
between two expressions in \dmllampat{} is given based on the concept of general
context. Later a reflexive and transitive relation $\leq_{dyn}$ on expressions in
\dmllampat{} is given which is equivalent to the \emph{operational equivalence} if the
right-hand expression of the relation are well-typed. Another way to put it,
two expressions in \dmllampat{} with relation $\leq_{dyn}$ are operational equivalent
if the expression on the right-hand side of the $\leq_{dyn}$ relation is well-typed.

Secondly, the generic type index language \dmlL{} introduced by the paper is a
pure simply typed language, in which
constraint relations can be properly defined. Types in \dmlL{} are called 
sorts for clarity while expressions in \dmlL{} is called terms, 
which are used to build types in \dmllamall.

Thirdly, the explicitly typed language \dmllamall{} is an extension of
\dmllampat{} with both universal and existential dependent types. Similar to
that in \dmllampat{}, dynamic
semantics of expression in \dmllamall{} is given based on 
evaluation context and redex. Formal proof
is provided for the type soundness of \dmllamall. The paper defines an
eraser operation, which maps expressions and types in \dmllamall{} into the
their counterparts in \dmllampat. It also illustrates the relation between
the dynamic semantics of a well-typed expression in \dmllamall{} and the
dynamic semantics of its erasure in \dmllampat, which makes it reasonable to
view these two dynamic semantics as the same.

Finally, the paper presents an external language \dmlzero{} together with a mapping
from \dmlzero{} to the internal language language \dmllamall. The process of such
mapping from an expression in \dmlzero{} to an expression in \dmllamall{} along with
its type is called elaboration. Elaboration rules are given to illustrate this
process. Erasure is defined to map an expression in \dmlzero{} into \dmllampat. The
dynamic semantics of the later can be viewed as the dynamic semantics of the
former. The paper proves that if an expression in \dmlzero{} can be elaborated into
an expression in \dmllamall, then their erasures in \dmllampat{} are operational
equivalent. Simply put, by the elaboration, we get an an expression with more
precise type while maintaining the same dynamic semantics.  

The design of the whole system introduced in the paper benefits both builders of
programming language and programmers using the language in the following
aspects.

The form of dependent
types studied in this paper is called a restricted form of dependent types,
which is substantially different from the usual form of dependent types in
Martin-L\"of's development of constructive type theory. One character of such
restricted form of dependent types is that the type index language
\dmlL{} used in such dependent type system doesn't contain any side
effect (e.g. recursion), which makes it practical for
programmers to reason about the types they want to use. Also it's practical
to build the type checker which is able to compare types within such system.

Though \dmllamall{} provides a way to exploit the dependent type system, 
which can facilitate program error detection and
compiler optimization, programmer may be quickly overwhelmed with the
demanding work to write a well-typed expression in \dmllamall{} as well as the
amount of effort for manual type annotation of the program. The introduction
of \dmlzero{} offers a solution to such problem. In \dmlzero{} programmer can write the
program with relatively simpler syntax and only need to provide a reasonably
small amount of type annotation in practical programming. The compiler can
then translate the program into \dmllamall{} while maintaining the dynamic
semantics. The illustration of such translation process and the formal proof
of the soundness of such translation constitute the main contribution of
the paper.

Going still further, the paper also extends \dmllamall{} with 
parametric polymorphism, exception
and references, attesting the adaptability and practicality of such approach
to supporting the use of dependent types in the presence of realistic
programming features.


\newpage
\section{Summary of ``Linear types can change the world!''\cite{Wadler1990Linear}}
  \label{section:lineartypes}

In traditional functional programming languages, objects can be created explicitly
and discarded implicitly. And as such, the ``update'' of an object is conveyed by
creating a new object and discarding the old. In such programming paradigm,
objects of the same value are treated as of the same identity, 
which doesn't actually match the computation model provided by real
physical machines in which objects are distinguished by their innate identities.
On one hand, such programming paradigm releases the programmers from the heavy duty of
tracking all the objects within the computation system, which is mostly
cumbersome and error-prone. On the other hand, it also deprives the
programmers of the power to manipulate objects directly and the general
mechanism for tracking ``real'' objects in the system (e.g. reference counting and garbage
collection) often leads to an inefficient implementation.

This paper presents a type system, in which types are divided into two families.
Objects of linear type have exactly one reference to them, and so require no garbage
collection. Objects of nonlinear type may have many pointers to them, or none, and
do require garbage collection.  Such system not only offers the explicit 
manipulation of objects with guaratantee of correctness but 
also retains the same ability of implicit objects
management as normal functional programming language. 

Linear type system applied in the paper is based on the linear logic
of J.-Y. Girard. The key idea in such system is that each assumption in the context of a
type judgement must be used exactly once. It cannot be duplicated or discarded.
Another way to put it, the typing rule of the system guarantees that a well
typed program can have only one reference to any object of linear type
and cannot discard such object implicitly.
With such constraints, it's sensible in such linear system that
each operation that allocates an object is paired with exactly one operation
that deallocates that object. The allocation operations create objects to which
a single reference exists. This reference may never be duplicated or discarded,
so the deallocation operations act on objects to which they hold the sole
reference. In particular, after an application the storage occupied by the function
(closure) may be reclaimed, which leads to the fact that a linear function
(closure) in such system can and must be applied only once.

The non-linear part of the type system presented in the paper is similar to that
of the simply-typed $\lambda$-calculus with the support of data type. 
Complex types comprising of both linear and non-linear types can be constructed while
the type system guarantees that non-linear data structure must not 
contain any linear components. This is sensible that the non-linear object,
which can be duplicated, can cause the linear object embedded inside to be
accessed once for each duplication.

Going still further, the system provides a
special ``let'' construct in which multiple references to an object is
allowed so long as the object is being ``read'' as indicated explicitly by the
programmers. The type system and the evaluation rule guarantee that all
those multiple references would be relinquished till only one left for the
coming ``write'' access.

Though the type system used in this paper is monomorphic, it is not difficult to
extend it to a polymorphic language with explicit type applications.
Besides, the language provided by the paper is in essence the traditional
$\lambda$-calculus
extended by pattern matching and data type declaration, which makes it a
good candidate to be incorporated into existing languages with more
advanced types.

\newpage
\section{Summary of ``Safe Programming with Pointers through Stateful Views''
  \cite{Zhu2005Safe}}
  \label{section:statefulviews}

ATS is a programming language with a type system rooted in a framework
\emph{Applied Type System}, which is derived from
\emph{DML}\footnote{Please refer to section \ref{section:DML} for details.}.
It supports a restricted form of dependent types, by which sophisticated
structures can be encoded with more confidence of correctness.
This paper illustrates another
novel and desirable feature of it, the support for safe programming with
pointers through a notion of \emph{stateful views}.

The linear type system in section \ref{section:lineartypes} provides a solution
to control the access to individual objects within the system. The paper takes
advantage of intuitionistic linear logic from another perspective. Instead of
introducing linear types for concrete objects in memory, it introduces linear
``views'' for abstract ``proof'', which depicts not only the linearity of the 
objects but also the locations of them.
Such feature is made possible by the introduction of a special \emph{sort}
\emph{addr} in the statics of ATS, which represents the address within the
memory. Through dependent types, a view is dependent on multiple addresses
within the memory, which in turn states the layout of the memory. Therefore,
such method is more flexible than that in section \ref{section:lineartypes}.

The type system of ATS is divided
into two families: nonlinear types and views. From the perspective of
isomorphism, a view corresponds to a logic proposition, its instance is an 
abstract object corresponding to the proof of the proposition. Hence, we just
call the instance of a \emph{view} a \emph{proof}. Such proofs are only used at
compile-time for performing type-checking and they are neither needed nor
available at run-time.

Formally, two forms of constraints including $\Sigma; \overline{P} 
\models P$ (persistent) and $\Sigma; \overline{P}; \overline{V} 
\models V$ (ephemeral) are formalized in ATS, which are needed to
define type equality. The rules for proving such constraints are given based on
intuitionistic logic and intuitionistic linear logic. Besides, for each
user-defined view constructor, a couple of constructor-related rules are
introduced. With such rules as well as those rules related to type judgement and
subtype relation, reasoning about the changes of the states of the memory is
turned into a type checking issue.

To sum up, the usage of \emph{stateful views} provides a light-weighted formal
method for reasoning about pointers and memory. Though extra effort is
needed to encode such reasoning explicitly in program, the manifist of such
reasoning makes it practical to apply the method to general sophisticated programs where
fully-automatic verification is infeasible.



\newpage
\section{Summary of ``Proofs and Types''
  \cite{Girard1989Proofs}}
  \label{section:proofstypes}

\subsection{Normalization}
A term in $\lambda$-calculus is called a normal form if it contains no redex. A
normal form is the termination of the reduction path. Starting from a term,
there are many reduction paths which may or may not lead to a normal form.
Typed $\lambda$-calculus behaves well computationally in the sense that each
term can be reduced to a unique normal form along any reduction path. Such property
guarantees that the program written in typed $\lambda$-calculus shall terminate
no matter which evaluation strategy is chosen.

The proof of such property consists of two parts: the uniqueness and existence of 
the normal form.

The uniqueness comes from that $\beta$ reduction is \emph{Church-Rosser},
which means that if $t \rightsquigarrow u, v$, then $\exists w. u, v
\rightsquigarrow w$. Its proof can be found in chapter 3.2 of
\cite{Barendregt1984Lambda}.

The existence of the normal form can be interpreted in two ways. One way
is that there exists a reduction path leading to the normal form. The other is
that all the reduction paths lead to the normal form. The former property is
called weak normalization, while the later is called strong normalization.

The proof of weak normalization is kind of construction proof in the sense that
it provides the algorithm for choosing at each step along the reduction path
an appropriate redex which leads us to the normal form. Since the algorithm is
defined recursively, an appropriate index has to be chosen to prove that the
algorithm terminates, which in this case is the degree of redex. At each step,
we locate in the term the redex with the maximal degree $n$ and we also require that
the subterm of the redex we pick cannot contain redex of the same degree. It
is then proved that after the reduction of the redex we pick, no new redex with
degree $\ge n$ would be created and also the number of redex with degree $n$ is
decreased by $1$. It's easy to reason that such algorithm would lead us to a
term without redex (normal form).

The proof theoretic technique for the strong normalization of simply typed 
$\lambda$-calculus can be used to prove the strong normalization 
of System T and System F, which
include a type of integers and hence codes Peano Arithmetic. Therefore, we have
to use a strong induction hypothesis, for which an abstract notion called
\emph{reducibility} is introduced. For simply typed $\lambda$-calculus, a set
$RED_T$ (``reducible terms for type T'') is defined by induction on the
\emph{type} T.
\begin{enumerate}
\item For $t$ of atomic type $T$, $t$ is reducible if it is strongly
normalisable.
\item For $t$ of the type $U \rightarrow V$, $t$ is reducible if, for all
reducible $u$ of type $U$, $t u$ is reducible of type $V$.
\end{enumerate}

The following three properties of reducibility are proved together by induction.
\begin{enumerate}
\item (CR 1) If $t \in$ RED$_T$, then $t$ is strongly normalisable.

\item (CR 2, Forward Property) If $t \in$ RED$_T$, and $t \rightsquigarrow t'$, then $t' \in$
RED$_T$.

\item (CR 3, Backward Property) If $t$ is neutral (substituting $t$ for any
free variable in any term $p$ doesn't create extra redex besides those which already
exist in $t$ and $p$), and whenever we convert a redex of $t$ we obtain a term
$t' \in$ RED$_T$, then $t \in$ RED$_T$.

\end{enumerate}

Finally we prove that all terms are reducible, which leads to the fact that all
terms are strongly normalisable. To prove this by induction, we strengthen the
induction hypothesis to handle the case of abstraction and come up with the
following proposition, from which we can get the ultimate goal by putting $u_i =
x_i$.\\
Let $t$ be any term, and suppose all the free variables of $t$ are among $x_1,
\dots,x_n$ of types $U_1, \dots, U_n$. If $u_1, \dots, u_n$ are reducible terms
for types $U_1, \dots, U_n$ then $t[u_1/x_1, \dots, u_n/x_n]$ is reducible.


\subsection{System F}

System F is a typed $\lambda$-calculus that differs from the
simply typed $\lambda$-calculus by the introduction of a mechanism of universal
quantification over types. Formally, system F contains the following scheme for
forming terms: if $v$ is a term of type $V$, then we can form
$\Lambda X.v$ of type $\Pi X.V$, as long as the variable $X$ is not free in the
type of any free variable of $v$. As such, the types in system F correspond to
propositions quantified at the second order.

System F allows recursive constructions to be embedded in a natural manner. For
a structure $\Theta$ described by a finite number of constructors $f_1, \dots,
f_n$ respectively of type $S_1, \dots, S_n$, its type in system F is 
$T=\Pi X.S_1 \rightarrow S_2 \rightarrow \dots \rightarrow S_n \rightarrow X$. Recursivity
is manifested by the fact that $X$ may appear within one of the types $S_i$. The
type $S_i$ must itself be of the particular form $S_i = T^i_1 \rightarrow T^i_2 \rightarrow \dots T^i_{k_i} \rightarrow X$. Then the
constructor $f_i$ can be encoded in system F as follows \\
$f_i = \lambda x_1^{T^i_1}. \dots \lambda x_{k_i}^{T^i_{k_i}}. 
\Lambda X. \lambda y^{S_1}_1. \dots \lambda y^{S_n}_n. y_i t_1\dots t_{k_i}$
where 
\begin{itemize}
\item  $t_j = x_j$, $X$ doesn't occur in $T^i_j$
\item  $t_j = x_j X y^{S_1}_1 \dots y^{S_n}_n$, if $X$ occurs positively in
$T^i_j$.

\end{itemize}

Based on such scheme, both simple types (e.g. booleans, product type, and sum
type) and inductive types (e.g. integer, list, and tree) can be encoded in
system F. The intuition behind such scheme goes as follows. Terms of inductive
types are encoded as high order functions with ``constructors functions'' as 
the input arguments. For example, the integer type is encoded as follows \\
$Int \overset{def} = \Pi X.X \rightarrow (X \rightarrow X) \rightarrow X$
\\ and the integer n is represented by \\
$\bar{n} = \Lambda X.\lambda x^X.\lambda y^{X \rightarrow X}.
\underbrace{y(y(y\dots(y}_\text{n occurrences} x)\dots))$ \\
These ``constructors'' are applied inductively based on the definition of the
structure corresponding to the type. Such form makes it easy to
define \emph{iterator} on terms of induction types in system F. However it's quite
difficult to define recursion, for which generally, a full iteration has to be taken
starting from the most ``atomic'', then reconstructs a new term (usually a
tuple) iteratively and uniformly. Starting from recursion, all kinds of common operations
related to those inductive types can be defined in system F, though the encoding
is unacceptably tedious with very low efficiency in evaluation.

Though it's not practical to program directly in system F
due to low efficiency in both coding and execution, 
system F serves as a formalization of the notion of parametric
polymorphism in programming languages, and forms a theoretical basis for
languages such as Haskell and ML.



\newpage 
\section{Summary of ``The Temporal Logic of Programs''
  \cite{Pnueli1977Temporal}}
  \label{section:LTL}
This paper provides a unified approach to program verification, which applies
to both sequential and parallel programs.

To verify a program, first we need to define what a program is. For that,
this paper presents a general framework to support the modelling of both
sequential and concurrent programs. Informally, a program can be viewed as
multiprocessor computer with shared memory, in which each processor runs its
statements independently. But each time only one processor can execute
one statement. The scheduling choice of the next processor to be stepped
is nondeterministic.

More formally, a dynamic discrete system including both sequential and
concurrent program consists of $<S, R, S_0>$ where 
\begin{itemize}
\item
$S$: the set of states the system may assume; 
\item
$R$: the transition relation holding between a state and
its possible successors, $R \subseteq S \times S$; 
\item
S0: the initial state.
\end{itemize}
The state of a system is represented by $s = <\pi_1, \dots , \pi_n; u>$. 
Each $\pi_i$
may be considered as the value of the program counter for
the i-th processor, while $u$ is
the shared data component.

In such model, the execution of statements on one processor can only
influence the other processor though the usage of shared data. But the user
can form up appropriate “atomic” statements based upon which program
is built. Therefore it’s still feasible to model in the framework the
execution of a concrete program in which interference between different
phases of concurrent statements’ execution may exist.

Based on the aforementioned framework, this paper illustrates how
to specify properties of systems and their development in time. First, relations
on states $q(s)$, expressed in a suitable language as well as explicit time
variables $t$ are introduced. Then the properties of a program is formed as the
development of the properties $q(s)$ in time $t$ (a.k.a. time functional $H(t,
q) \equiv q(s_t))$, which can be interpreted as “at time instance $t$, the state
of the system (denoted by $s_t$) satisfies the predicate $q(s)$”.

Though any arbitrary complex time specification can be expressed in this
way, the paper focuses on two major categories of them, invariance and
eventuality (temporal implication). The former one is a statement with
only one time variable which is universally quantified. It is in the form
 $\forall t. H(t, q)$.  The later one is a
statement of the form 
$\forall t_1. \exists t_2. (t_2 \geq t_1) \land (H(t_1, \varphi) \rightarrow
H(t_2,\psi))$, and we use $\varphi \rightsquigarrow \psi$ to denote such form.

Three proof approaches (invariance, well founded
set, and temporal reasoning) are presented in the paper for the reasoning of
both invariance and eventuality properties. The basic idea of the third 
approach, focused by the paper, is that one derives simple time dependency
relations (e.g. eventuality) directly from the system transition rules ®
and then use combination rules, and general logic reasoning to derive more
complex properties.

Focusing on establishing simple properties including invariance and
eventuality, the paper chooses to find a minimal basis for temporal reasoning
without taking the brute force approach of installing an explicit real time
clock variable. Based on such design principle, two formalizations of the
temporal reasoning are presented in the paper. First one is an axiomatic
system (ER) with the eventuality connective $\rightsquigarrow$. 
Without time variable $t$,
the statement $\varphi \rightsquigarrow \psi$ is interpreted as that for all executions,
if $\varphi$ is valid at certain moment, then $\psi$ would be valid as well
later but eventually.

Though system ER is proved to be sound and complete, it’s very difficult to
express natural intuitive arguments for the behavior of concurrent programs in
(ER). Therefore, the paper presents a second formalization, which actually
adopts a fragment of the tense logic $K_b$. Two basic tense operators, $F$ and
$G$ are introduced. Denoting the present by $n$, they can be interpreted as
follows:
\begin{itemize}
\item
F(p): $\exists t. t \geq n \land H(t,p)$

\item
G(p): $\forall t. t \geq n \rightarrow H(t,p)$

\end{itemize}

The $K_b$ fragment has the following properties. Firstly, since
$G(p \rightarrow F(q))$ matches the notion of eventuality, 
the $K_b$ fragment is at least
as strong as (ER). Secondly, as a fragment of $K_b$, the system is proved to be
complete. Finally, the $K_b$ fragment is isomorphic to the modal logic system
$S_4$ with $G$ standing for $\Box$ and $F$ standing for $\Diamond$. Therefore the $K_b$
fragment presented in the paper is decidable.

A scheme of a proof in such logic system consists of two separate phases. In
the first phase, one reasons about states, immediate successors and their
properties to translate all the relevant properties of the program into
basic tense-logic statements. This is done via the application of “domain
dependent” axioms, which restricts the future to only those developments
which are consistent with the transition mechanism of the system. The
provision of such axioms in the logic system makes it distinguished from the pure
tense logic $K_b$. In the second phase, one uses the pure logic rules in tense
logic and manipulates those tense logical statements into the final result.

Going still further, the paper points out that the validity of any arbitrary tense formula
on a finite state system is decidable and can be proved in an extension to the $K_b$ fragment
presented in the paper.

In summary, the $K_b$ fragment presented in the paper can be viewed as
a simplified version (with less operators) of Linear-Time Temporal Logic
(LTL) illustrated in \cite{Huth1999Logic}. Users' natural intuitive reasoning of programs
can be encoded in such systems in a formal and practical way. As a trade
off, the expressive power of such systems is limited due to the lack of time
variable. However, quite a lot of properties of time in practice can already
be captured in such system, which leads to the growing application of LTL
in the field of model checking.

The approach taken in the paper can be classified as endogenous approach,
in which we immerse ourselves in a single program which we regard as the
universe, and concentrate on possible developments within that universe. With
such special universe, it's natural trying to find a semantical proof
(by model checking), which seems to be a more practical and economic way
than trying to find a syntactical proof. I think that's another reason
why LTL is so widely used in the field of model checking.

However, to provide tools and guidance for constructing a correct system
rather than just analyse an existing one, we have to take exogenous approaches,
which usually suggest a uniform formalism dealing with formulas with varying
context (program segment). Due to the complexity of such approaches, no
dominant system has been provided to fulfill such tasks, especially for the
concurrent programs.

\newpage

\section{Summary of ``The Theory and Practice of 
  Concurrency''\cite{Roscoe1997Theory}} 
  \label{section:CSP}

Communicating Sequential Process (CSP) is a formal language for describing
interactions between different components within a system on the level of
communication. It was introduced by Hoare in the late 1970s. And since then,
it has been widely developed and applied in the specification and verification
of reactive and concurrent systems in which synchronization and communication
play a key role.

Components with independent inner states are modeled as \emph{processes} in
CSP. There is no shared information between the process and its environment. They
can only communicate with each other through the synchronization of
\emph{events}. Such events are both atomic and instantaneous. The set
of events that may be communicated by a process is said to comprise its
alphabet $\Sigma$. If a process offers an event with which its environment
agrees to synchronize, the event is performed. A finite sequence of events that
a process may perform is call a \emph{trace} of the process. For (semantic)
convenience the alphabet of each process is extended to contain two further
events: $\csptau \notin \Sigma$ represents an internal event and $\csptick \notin
\Sigma$ represents termination.


CSP is equipped with a rich set of process operators as well as several atomic
processes, based on which complicated processes can be defined recursively.
For each process operator, CSP also provides a set of algebraic
laws, by which we can reason about the equality of two processes.
These laws are chosen in such manner that if two processes are ``equal'' according
to the laws, their communicating behaviours are indistinguishable by the
environment. CSP is thus categorized as a member of process algebras. The
common process operators in CSP includes prefixing $\csppre{a}{P}$, external
choice $\cspext{P}{Q}$, internal choice $\cspint{P}{Q}$, parallel composition
$\csppara{P}{A}{Q}$, hiding $\csphide{P}{A}$, and sequential composition
$\cspseq{P}{Q}$. The process $\csppre{a}{P}$ offers its environment
the opportunity to synchronize on event $a$ in which case it
then behaves like $P$. The process $\cspext{P}{Q}$ offers its environment
a choice between $P$ and $Q$ based on synchronization with their initial
events respectively. The process $\cspint{P}{Q}$ behaves like either $P$ or
$Q$ but the choice is made internally, beyond environmental influence. The
process $\cspseq{P}{Q}$ behaves like $P$ and, if that terminates, then
behaves like $Q$. The process $\csphide{P}{A}$ executes the events in the
set $A$ internally, without synchronization by its environment; they can
be thought of as being replaced by $\csptau$ events. The parallel composition
$\csppara{P}{A}{Q}$ requires $P$ and $Q$ to synchronize on each event $a
\in A$, but performs other events of $P$ or $Q$ as determined by those
processes. The atomic processes includes $STOP$, $SKIP$, $DIV$. $STOP$ models
deadlock by offering no events. $SKIP$ models successful termination by
offering only $\csptick$. $DIV$ models the process that does nothing but
diverge, that is to enter into an infinite sequence of consecutive internal actions.

CSP has a range of semantic models (denotational semantics) including the traces models
$\csptracesmodel$, the stable failures model $\cspfailuresmodel$, and the
failures-divergences model $\cspfdrmodel$. In the traces model of CSP, a process
$P$ is represented by $traces(P)$ which is the set of all its possible traces. 
Such semantics is useful in deciding questions of safety. Going still further, the 
failures-divergences model is useful in specifying both safety and liveness
properties. In such model, a process $P$ is represented by two sets of
behaviours:
\begin{itemize}

\item $divergences$ are finite traces, each of which can lead the process to
diverge, the net effect of which is that the process won't accept anything
from the environment
\footnote{The precise definition of $divergences$ should contain any finite
sequence with prefix which can lead the process to diverge.
Such definition offers a more concise theoretic treatment of refinement
relation, yet matches our intuition that what a process can do after divergence
is of no concern. But since such precise definition contains no more useful
information about the behaviour of the process, most of time the simpler version
of definition is used to illustrate the idea of $divergence$.
}
;

\item $failures$ are pairs $(s, refusals)$ where $s$ is a finite trace of the
process and $refusals$ is a set of events it can refuse after $s$
\footnote{Due to the same reason as for the $divergence$,
precisely, we should use $failures_{\bot}$ instead of $failures$, whose definition
is $failures_{\bot}(P) = failures(P) \cup \{(s, X)| s \in divergences(P)\}$. 
}.
\end{itemize}
The failures-divergences model allows us to assert that a process must eventually accept
certain event from a set that is offered to it since refusal and divergence are the
two ways a process can avoid doing this, and we can specify in this model that
neither of these can happen. The stable failures model can be viewed as a simplified version of
failures-divergence model, only containing the information of failures. Though
not as powerful as the later, it is sometimes advantageous to use such model
since the calculations required to determine if a process diverges are very
costly.

With the aforementioned semantic models, \emph{refinement} relation between
processes can be defined as subset relation. Process $Q$ is a
refinement of process $P$ means that $Q$'s behaviors are contained in those of
$P$. Formally, for any of the three semantic models $\mathcal{M} \in
{\csptracesmodel, \cspfailuresmodel, \cspfdrmodel}$, we define such relation by
$P \sqsubseteq_{\mathcal{M}} Q \iff \mathcal{M}[Q] \subseteq \mathcal{M}[P]$
where $\mathcal{M}[P]$ denotes the semantics of process $P$ in semantic model
$\mathcal{M}$. Also we have that for each operator $\bigoplus$ in the language,
$P \bigoplus Q$ is a refinement of $P_0 \bigoplus Q_0$ as long as $P$ is a
refinement of $P_0$ and $Q$ is a refinement is $Q_0$ (with obvious modifications
for non-binary operators). In this way, CSP provides a theoretic background
for refinement development from abstract specification to implementation.

From my perspective, it's difficult to understand the concepts of CSP
without referring to its operational semantics as a tuition.
The operational semantics of a process is a \emph{labelled transition system} (LTS), 
which is a directed graph with a label on each edge. Each node can be viewed as
a state of the process and the labels on the leaving edges of the node represent
the actions the process can possibly take at that state. The set of
possible labels is $\Sigma^{\csptau, \csptick} = \Sigma \cup \{\csptau, \csptick\}$. 
For each CSP operators, we have several rules to derive the LTS of the
top-level process from the LTS' of its syntactic parts (sub-processes). 
The operational and denotational semantics of CSP are congruent in the sense
that we can extract all three kinds of denotational semantics of a process from its
LTS directly. Such congruence makes it possible to build model checkers for
refinement checking of two processes practically.

All three refinements are supported by the automatic refinement checker FDR
which proves or refutes assertions of the form $P \sqsubseteq_{\mathcal{M}} Q$.
FDR inputs processes expressed in \cspm, which is a standard for
machine-readable CSP. \cspm{} expresses CSP by a functional language, offering
constructs such as \emph{function}, \emph{let} expressions and supporting
pattern matching. It also provides a number of predefined data types, including
booleans, integers, sequences and sets, and allows user-defined data types to
certain extent. In short, with the support of \cspm, the computation of 
sub-process objects (e.g. events and process parameters) can be expressed along
with the CSP process.  Besides FDR, some other tools such as ProB and PAT can
do refinement checking as well. Furthermore, these tools can also do 
LTL model checking of CSP model encoded in languages specific to these tools.




\newpage
\section{Summary of ``Model-Checking CSP''\cite{Roscoe1994Modelchecking}}
  \label{section:modelcheckingcsp}
The denotational semantics of processes in CSP equips us with a mathematical
foundation for defining a special relation between processes called
``refinement'' relation. As the co-worker of the inventor of CSP, the author of
the paper describes his work of building a model-checker/refinement checker for
CSP based on its operational semantics, experience of the application of such tool,
 and prospects about the improvement of it.

The standard model for CSP is the \emph{failures/divergences} model. Such model
($\cspfdrmodel_{\Sigma}$) over a given finite alphabet $\Sigma$ of communication
is the set of pairs of sets $(F, D)$ satisfying certain ``healthiness'' properties
\footnote{Please refer to section \ref{section:CSP} for the precise definition of the
failures/divergences model.}
. The denotational semantics of a process $P$ is then an element $(F_P, D_P)$ of
$\cspfdrmodel_{\Sigma}$. 
Based on these, the refinement relation $P \sqsubseteq_{\cspfdrmodel} Q$
\footnote{For the purpose of clarity, we may use the notation of $\sqsubseteq$
instead since we only focus on the failures/divergences model in the paper.}
is defined by $F_Q \subseteq F_P \land D_Q \subseteq F_P$.
Due to the congruence between the denotational semantics and operational
semantics of a process. The decision problem of whether 
$P \sqsubseteq Q$ can be carried out by their operational semantics.

This paper only tackles the problem in which the operational semantics of the
processes on both sides of the relation are of \emph{finite state}. This simply
means that, as the operational semantics is unfolded, the resulting LTS contains
only finite nodes. Also the paper doesn't take into consideration the
termination of process, which means that the LTS' being worked on are finite
directed graphs where all edges are labelled with an action, either $\tau$ or
visible. But it's not difficult to add the support of $\csptick$ to the method
developed in the paper.

The basic idea of refinement checking is that, given a certain trace $s$ and a
reachable state in the implementation process (right-hand of the refinement
relation), there is a corresponding state in the specification process
(left-hand of the refinement) reachable on the same trace and has the same
behaviour.

Typically, the LTS arising from CSP descriptions contains a high degree of
nondeterminism, in the sense that after any trace $s$ of visible actions there may
be many states of a system which the process might be in. This can happen both
because of the existence of invisible actions and because of the branching that
occurs when a node has two identically-labelled actions, whether visible or
invisible. Any method for deciding refinement between these systems will have
to keep track of all the states reachable at the specification side on a given
trace $s$. The method provided in the paper accomplishes this by normalizing the
specification process before doing the refinement checking. The normalization
is processed in two stages. In first stage, multiple nodes in the LTS, which 
are reachable after certain trace, are merge into one node in the new LTS. And
the new node conveys all the information of the original nodes, namingly all the
$refusals$ sets of the original nodes. Also if one of the nodes being merged
diverges in the original LTS, then the new node is marked diverging in the new
LTS. Such strategy matches our idea that if certain trace can lead the process
to diverge, then whatever after is of no concern. Second stage is actually a
bisimulation checking, in which bisimilar nodes are merged together.

After normalization, the refinement checking is conveyed by enumerating and
checking all the
possible pairs $(v, w)$ of nodes satisfying the following condition: $\exists$
traces $s$, such that $v$ is reachable after $s$ in
the normalized specification process and $w$ is reachable after $s$ in
the implementation process. For the refinement checking in the
failures/divergences model, the checking on each pair includes the compatibility
of initial actions, divergence, and refusals. If we only check the initial
actions of the nodes in the pair, then this method is degraded into a refinement
checking in the traces model. Normally the enumeration of pairs is done by DFS
(Depth First Search). If the checking on certain pair fails, then the whole
refinement relation doesn't hold. The path leading to such pair serves as a
counter-example. BFS (Breadth First Search) can then be applied to find the
shortest path leading to an error.

A tool called FDR was built to do the refinement checking based on the method
shown above with the specification and implementation processes encoded in
the \cspm{} language. Besides the syntax of \cspm, FDR also puts some extra 
syntactic restrictions on the input processes to enforce finite state space.
However process with infinite state space can still be created under these
restrictions due to the use of recursion and infinite data types. It's the
users' responsibility to treat such processes with care.


Though checking failures/divergence refinement of LTS is PSPACE-hard,
fortunately ``real'' process definitions simply do not behave as badly as some
pathological examples. But great effort has to be taken when building FDR due 
to the efficiency concern. The paper discusses the usage of hashmap and sorted
list for implementing set in different parts of FDR and justifies such choices.
Going still further, the paper discusses the possibilities to improve
FDR to handle more practical usage. Three possibilities I think more applicable
goes as follows. First one is to build certain
logical inference tool based on the algebraic rules of CSP to alleviate the work
of the model-checking. Second one is to compress the state-space so
that we can prove results of the state space in blocks instead of expanding the
state-spaces of processes fully and explicitly. The third one is to include
symbolic representations of data in the LTS instead of expanding the LTS for each
possible values. Such method is applicable to CSP processes in which data (at
least most of the data) does not alter the control-flow of a process. Some work 
related to this topic has been done in \cite{Lazic1999Semantic}.

\newpage 
\section{Summary of ``How to Make FDR Spin --- LTL Model Checking of CSP by
Refinement''\cite{Leuschel2001How}}
  \label{section:FDRSPIN}

In ``classical'' model checking, systems to be checked are encoded in the
specification language used by the model checker. Extracted from these
specifications, models of the systems are then checked against the expected
correctness properties, which are encoded in certain formal language, by the
model checker. Among all those formal languages, Linear-time Temporal Logic 
(LTL) allows the specification of temporal properties encountered in practice
 in a natural and succinct manner, which makes LTL model checking a common
approach for the verification of hardware and software system.

The concept of refinement relation in Communicating Sequential Process (CSP)
offers a new approach for model checking. In such method, system to be checked is
specified as a process (implementation process) in CSP, while the correctness 
properties are also specified as a process (specification process) in CSP. The
verification of correctness properties of the system is then turned into the
question whether the implementation process (or process derived from it) is a
refinement of the
specification process, which can be decided by the refinement checker (e.g.
FDR). This refinement-based approach suits itself very nicely to the stepwise
development of systems. However there is no general way to
generate the specification process and the appropriate derivation of the
implementation process for any correctness properties, which makes such approach
less usable in practice. To tackle this, the paper presents a new approach which is also based
on refinement checking but can check the implementation process against the
correctness properties specified in LTL directly and mechanically. 

In ``classical'' LTL model checking, that a system satisfies an LTL formula $\phi$ 
means that all of its computations satisfy $\phi$ and a computation is defined as an
infinite sequence
of states, the transitions of which are possbile in the system. Similar concept is applied to
CSP process in this paper. That a process satisfies an LTL formula $\phi$ means that
the process cannot deadlock and all of its infinite traces satisfy $\phi$. 
And an infinite trace of a process is an infinite sequence of events that
the process can possibly communicate with. Given an infinite trace $\pi =
\pi_0,\pi_1,\dots$. We define $\pi^i$ to be suffix of $\pi$ starting from
$\pi_i$. $\pi \models \phi$ (a trace $\pi$ satisfies $\phi$) is
defined as follows:
\begin{itemize}
\item[-] $\pi \not\models false$
\item[-] $\pi \models true$
\item[-] $\pi \models a$ iff $\pi_0 = a$
\item[-] $\pi \models \neg a$ iff $\pi_0 \neq a$
\item[-] $\pi \models \phi \land \psi$ iff $\pi \models \phi$ and $\phi \models \psi$
\item[-] $\pi \models \phi \lor \psi$ iff $\pi \models \phi$ or $\phi \models \psi$
\item[-] $\pi \models X\phi$ iff $\pi^1 \models \phi$
\item[-] $\pi \models \phi\ U \psi$ iff there exists a $k \geq 0$ such that
$\pi^k \models \psi$ and $\pi^i \models \phi$ for all $0 \leq i < k$
\item[-] $\pi \models \phi\ R\ \psi$ iff for all $k \geq 0$ such that $\pi^k
\models \neg \psi$ there exists an $i$, $0 \leq i < k$ such that $\pi^i \models \phi$
\end{itemize}

We denote by $|\phi|_\omega$ the set of infinite traces which satisfy the
formula $\phi$. It's difficult and sometimes impossible to come up with a
process ($Spec_\phi$) which can generate the same set of infinite traces
 as $|\phi|_\omega$. Even if we find
$Spec_\phi$, a traces refinement test $Spec_\phi \sqsubseteq_\csptracesmodel S$
is not adequate to model check $S \models \phi$ since refinement checking in FDR is based
on \emph{finite} traces only. Going further, the papers points out that though failures
refinement test can guaratantee the preservation of LTL properties as long as
$Spec_\phi$ is finite-branching, such condition doesn't hold in most cases
in practice.

One more problem with the previous definition of satisfaction of LTL formula
is that it overlooks the fact that a CSP process which may deadlock
is still possible to ``satisfy'' an LTL formula intuitively. To tackle this, the
paper introduces an extra event $\Delta$ which indicating deadlock. Any
deadlocking trace of the process $S$ is then turned into an infinite trace by
appending an infinite number of $\Delta$'s. To capture the intuition when a
deadlocking trace satisfies an ordinary LTL formula $\phi$ over $\Sigma$, we can
translate from an LTL formula $\phi$ into a formula $\phi_{\Delta}$ over $\Sigma \cup \{\Delta\}$,
e.g., $\phi\ U \psi$ is translated to $(\neg \Delta \land \phi)\ U \psi$.
A process $S$ satisfies a LTL formula is now defined as follows:
\\$S \models \phi$ iff $\forall \pi \in|S|_\Delta, \pi \models \phi_{\Delta}$
where $|S|_{\Delta} = |S|_{\omega} \cup \{\gamma\Delta^{\omega} | (\gamma,
\Sigma) \in failures(S)\}$.

By existing automaton theory, we can create a B\"uchi automaton $\mathcal{B}$
(from $\neg \phi_{\Delta}$)
over $\Sigma \cup \{\Delta\}$ such that $|\mathcal{B}| =
|\neg \phi_{\Delta}|_\omega$, where $|\mathcal{B}|$ is the set of accepting words of
$\mathcal{B}$. With the aforementioned setting, the question whether a process
S satisfies the LTL formula $\phi$ ($S \models
\phi$) is now turned into the question whether $|S|_{\Delta} \cap |\mathcal{B}| 
= \phi$. It's easy to see that if the answer is yes, then $S \models \phi$, otherwise $S
\not\models \phi$.

At this stage, we have two possible strategies. First is to create another
B\"uchi automaton $\mathcal{A}$ whose set of accepting words is $|S|_{\Delta}$
and then use existing algorithm to check the emptiness of the intersection of
the sets of accepting words of these two B\"uchi automata (a.k.a checking the
emptiness of $\mathcal{A}
\cap \mathcal{B}$). Research following this path for LTL checking CSP 
can be found in \cite{Sun2009Model}.

This paper takes the second strategy in which a process $TESTER$ is generated from the B\"uchi automaton
$\mathcal{B}$ and the decision problem is then turned into several refinement
checks based on the parallel composition of $TESTER$ and $S$. The intuition
behind this is that $TESTER$ ``represents'' the traces accepted by the B\"uchi
automaton $\mathcal{B}$, and we want to check whether the process $S$ has
``similar'' traces to that of the $TESTER$. If it has, then $S$ has a trace $\pi
\models \neg \phi$ which indicates that $S \not\models \phi$. Concretely, the paper 
describes the method as follows.

Noticing that the B\"uchi automaton $\mathcal{B}$ may accept certain traces
(in which events in $\Sigma$ appear after $\Delta$) which no CSP process 
can possibly generate, this paper translates $\mathcal{B}$ into a simplified
automaton called
B\"uchi $\Delta$-automaton $\mathcal{B}_\Delta$ whose accepting set of words
$|\mathcal{B}_\Delta| = |\mathcal{B}| \cap (\Sigma^\omega \cup
\Sigma^*.\Delta^\omega)$. 
The process $TESTER$ can be generated from $\mathcal{B}_\Delta$
with the introduction of three extra events $success$, $deadlock$, and $ko$.
And to check $S \models \phi$, the following two checks shall be performed.
\begin{enumerate}

\item $SUC \sqsupseteq_\csptracesmodel 
  (\csppara{S}{\Sigma}{TESTER}\backslash(\Sigma \cup \{deadlock, ko\})$ 
  where $SUC = \csppre{success}{SUC}$

\item $\csppre{deadlock}{STOP} \sqsupseteq_\cspfailuresmodel 
       (\csppara{S}{\Sigma}{TESTER}(\Sigma \cup \{success\})$ 

\end{enumerate}
By detailed discussion, the paper shows that either of two checks succeeds iff
$S \not\models \phi$.

This paper uses FDR refinement checker to perform the refinement checking
involved in the process, which means that optimisations such as hierarchical
compression, data-independence and induction can be applied. But it's unclear
whether the overall system has better performance over the mothod developed
following the first possible strategy.
Another issue is that since FDR doesn't provide any
``candidiate'' trace when the refinement check succeeds, 
which indicates that $S \not\models \phi$, there is no way to get the
counter-example needed by the user to reason about the cause of the failure.


\newpage

\section{Summary of ``Polymorphic CSP Type Checking''\cite{Gao2001Polymorphic}}
  \label{section:CSP_Type_Checking}

This paper illustrates the design of a type checker for CSP$_M$, which is based
on Milner's type inference algorithm and Robinson's unification algorithm.

To begin with, this paper explains the classical type inference algorithm, which
is similar to the one I implemented for the CS 552 Compiler course. Then it
introduces two categories of types in addition to types usually associated with
functional languages (e.g. \emph{int} and \emph{bool}): unqualified types and qualified types.
Unqualified types including \emph{process}, \emph{event} function type, product
type, sequence type, set type, union type (I think this one is similar to
datatype in ATS.), dot type (unique to \cspm{}'s dot composition syntax) and
yield type (type for channels and tags of datatype). Qualified types are types
subject to additional constraints. There are only four constraints, which indicate
that certain type must have the following four kinds of properties: 1. the type
must supports equality operation; 2. the type supports ordering comparison; 3.
the type is actually a well-formed function type; 4. the type must be a process
type. (Compared to ATS, such constraints are relatively simple.)

The paper presents an extension of Robinson's unification algorithm so that it
can handle both the dot and qualified types. Going still further, a type
inference algorithm, which extends Milner's polymorphic type checking algorithm,
is presented.

I think I'd better acquire some knowledge about type unification and type
inference from textbook so that I can fully understand this paper.

\newpage
\section{Summary of ``Java2CSP: A System for Verifying Concurrent Java 
Programs''\cite{Shi2000Java2CSP}}
  \label{section:java2csp}
This paper presents the system \textbf{Java2CSP} which translates concurrent
Java programs into \cspbold{} processes, which can be then be fed to
model-checking tool FDR to verify the synchronization behaviour of the
Java program, such as deadlock and livelock.

\cspbold{} processes which simulate those mechanisms that are supported by Java
but not by \cspbold{} such as shared variables, threads and monitors are called
\emph{process patterns}.

The Java bytecode of a Java program is used as the source code for the
translation. The main reason I agree with is that ``A Java program can include
some auxiliary sources from libraries or from other users, which are sometimes
only available in their bytecode representations''.

One challenge is to provide CSP specification of corresponding mechanism in 
Java programs. For this, a set of CSP process patterns have been developed which specify
the general behaviour of a set of Java methanisms, like shared variables,
threads and monitores. Also the abstract reperessentation of Java object-model,
types, and algorithms have been formalized.
The author says algorithms are implemented to identify threads which can 
run concurrently, and to identify
variables which influence the behaviour of threads in different ways. 
But no further illustration in the paper at all.

Another challenge is to generate CSP model which can be analysized efficiently by
tools such as FDR.

\subsection{Quote}
The output of Java2CSP is again the input for FDR, the improvement of hte
quality of the output format with respect to the readability and the
recoverability is also an indispensable effort to make the system practical
useful for large Java programs.

\subsection{Comment}
The tool now only supports the verification of deadlock and livelock behaviours
of the Java programs. No support for the verification of LTL properties.



\newpage
\section{Summary of ``Model-Checking CSP-Z: Strategy, Tool Support 
  and Industrial Application''\cite{Mota2001ModelChecking}}
  \label{section:modelcheckingcspz}


\newpage
\section{Summary of various model checkers}
  \label{section:var_model_checker}

\subsection{SAL}
The fact that SAL is based on the symbolic approach to model checking and that
the symbolic technique is not suitable for software analysis where dynamic
creation of objects/data loses relevance in the context of applicative RAISE
specifications, where no dynamic object creation is possible.  Together with
this idea are the multiple advantages that the symbolic management of states
can offer: sets of states can be handled collectively (better temporal
performance in some cases) and the ability to handle much bigger state
spaces due to symbolic representation (compared with the explicit state
representation approach).\cite{Perna2005Model}

\subsection{List of model checker}
http://en.wikipedia.org/wiki/List\_of\_model\_checking\_tools

\subsection{Erlang}
Erlang is based on the ACTOR model, not CSP. The main practical differences
between Erlang and CSP is that Erlang uses asynchronous dynamically-typed
messages sent to a particular address (process id), whereas CSP systems usually
deal with synchronous messages sent down a particular, typed channel. But
they are both message-passing systems with the idea of removing shared mutable
data, as you say. For an implementation of CSP in the pure functional language
Haskell, see my library CHP (http://www.cs.kent.ac.uk/projects/ofa/chp/).
(From internet.)


\newpage
\section{Summary of ``An Analytical and Experimental Comparison of CSP 
Extensions and Tools''\cite{Shi2012Analytical}}
  \label{section:cspm_vs_csps}
\subsection{summary}
This paper compares \cspm{} and \csps{} from the aspects of syntax, operatioinal
semantics as well as their supporting tools such as \fdr{}, \prob{}, and
\pat{}. Such comparison can be used to guide users to choose the appropriate CSP
extension and verification tool based on the system characteristics.

I didn't find the information I am seeking about the roles of channels in
\csps{} when dealing with refinement checking. Do the messages passed through
channels constitute the events of refinement checking? I don't know.

4.1 of this paper summaries various optimization approaches used by different
model checkers. Also it compares the abilities of \fdr{} and \pat{} to do LTL
checking.

This paper includes several benchmarks for the comparison of the efficiency of
different model checkers.


\subsection{quote}
enhancing CSP by taking data and other system aspects into account

\csps{} integrates high-level CSP-like process operators with low-level
procedure code.

formally model concurrent systems, compositional operators, non-trivial data
structures, different verification capabilities

perspectives of modeling and verification needs, reasoning power of their
supporting tools, reason techniques and verifiable properties, the semantic
differences between \cspm{} and \csps{} lead to different state spaces and
optimizaitons in model checking

expression language


%%%%%%%%%%%%%%%%%%%%%%%%%%%%%%%%%%%%%%%%%%%%%%%%%%%%%%%%%%%%%%%%%%%%
\newpage
% 
% \bibliographystyle{apalike}
\bibliographystyle{plain}
\bibliography{alex_ren}
 

\section{Summary of ``Combining CSP and B for Specification and 
Property Verification''\cite{Butler2005Combining}}


\subsection{quote}
The state of a B machine is a mapping from variables to values, 
while the state of a CSP pocess is a syntactic process expression.


%%%%%%%%%%%%%%%%%%%%%%%%%%%%%%%%%%%%%%%%%%%%%%%%%%%%%%%%%%%%%%%%%%%%
\newpage
% 
% \bibliographystyle{apalike}
\bibliographystyle{plain}
\bibliography{alex_ren}
 
\end{document}



